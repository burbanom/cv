%!TEX TS-program = xelatex
%!TEX encoding = UTF-8 Unicode
% Awesome CV LaTeX Template for Cover Letter
%
% This template has been downloaded from:
% https://github.com/posquit0/Awesome-CV
%
% Authors:
% Claud D. Park <posquit0.bj@gmail.com>
% Lars Richter <mail@ayeks.de>
%
% Template license:
% CC BY-SA 4.0 (https://creativecommons.org/licenses/by-sa/4.0/)
%


%-------------------------------------------------------------------------------
% CONFIGURATIONS
%-------------------------------------------------------------------------------
% A4 paper size by default, use 'letterpaper' for US letter
\documentclass[11pt, a4paper]{awesome-cv}

\setdefaultlanguage{french}
\sethyphenation{french}{} % Add words between the {} to avoid them to be cut

% Configure page margins with geometry
\geometry{left=1.4cm, top=.8cm, right=1.4cm, bottom=1.8cm, footskip=.5cm}

\setmainfont{Lato}

% Set false if you don't want to highlight section with awesome color
\setbool{acvSectionColorHighlight}{false}

% If you would like to change the social information separator from a pipe (|) to something else
\renewcommand{\acvHeaderSocialSep}{\quad\textbar\quad}


%-------------------------------------------------------------------------------
%	PERSONAL INFORMATION
%	Comment any of the lines below if they are not required
%-------------------------------------------------------------------------------
% Available options: circle|rectangle,edge/noedge,left/right
%\photo[circle,noedge,left]{img/MB.png}
\name{Mario }{Burbano}
\position{Ingénieur et Analyste Cloud Data{\enskip\cdotp\enskip}Docteur en Chimie par modélisation numérique}
\address{91 rue du Colonel Fabien, 92160 Antony, France}

\mobile{+ 33 6 43 27 79 22}
\email{burbanom@tcd.ie}
%\homepage{www.posquit0.com}
\github{burbanom}
\linkedin{burbanom}
% \gitlab{gitlab-id}
% \stackoverflow{SO-id}{SO-name}
% \twitter{@twit}
% \skype{skype-id}
% \reddit{reddit-id}
% \medium{madium-id}
%% \firstname and \lastname will be used
\googlescholar{zPuB-z0AAAAJ}{Mario Burbano}
% \extrainfo{extra informations}

% \quote{``Be the change that you want to see in the world."}


%-------------------------------------------------------------------------------
%	LETTER INFORMATION
%	All of the below lines must be filled out
%-------------------------------------------------------------------------------
% The company being applied to
\recipient
  {Omega SA}
  {2502 Biel/Bienne\\Switzerland}
% The date on the letter, default is the date of compilation
\letterdate{\today}
% The title of the letter
\lettertitle{Watch Data Engineer}
% How the letter is opened
\letteropening{Madame, Monsieur,}
% How the letter is closed
\letterclosing{
Je vous remercie de l'attention que vous porterez à ma candidature et j'espère que vous me contacterez afin d'en apprendre plus sur le poste ouvert et que je puisse vous partager mes expériences qui vous seraient profitables. 
Veuillez croire, Madame, Monsieur, en l’expression de mes sentiments les plus sincères.\\ \\ Bien cordialement,}
% Any enclosures with the letter
%\letterenclosure[Attached]{Curriculum Vitae}


%-------------------------------------------------------------------------------
\begin{document}

% Print the header with above personal informations
% Give optional argument to change alignment(C: center, L: left, R: right)
\makecvheader[R]

% Print the footer with 3 arguments(<left>, <center>, <right>)
% Leave any of these blank if they are not needed
\makecvfooter
  {\today}
  {Mario Burbano~~~·~~~Lettre de motivation}
  {}

% Print the title with above letter informations
\makelettertitle

%-------------------------------------------------------------------------------
%	LETTER CONTENT
%-------------------------------------------------------------------------------
\begin{cvletter}

%\lettersection{Introduction}
\begin{comment}
Description du poste
Analyser les données horlogères et mettre les résultats sous forme visuelle --> articles scientifiques
Participer activement à l’amélioration et à l’innovation des produits --> Je suis passionné par la mise en œuvre de systèmes de traitement de données et par la façon dont ils peuvent faciliter le processus de prise de décision basée sur les données.
Rechercher et développer des outils de caractérisation et d’aide à l’analyse des données de contrôle
Participer à la mise en place des structures de données de contrôle des montres et des mouvements
Créer les standards de données horlogères du département
Construire des plans d’expérience pour vérifier certaines données
Gérer des projets d’amélioration de la remonté des données

Profil
Formation de Technicien avec expérience dans l’analyse des données ou formation universitaire dans le domaine de l’ingénierie ou de l’informatique (Ingénieur EPFL)
Expérience en gestion de projets
Maîtrise des logiciels self BI comme Power Bi
Connaissances des outils ETL
Expérience confirmée en SQL et en programmation
Passionné par les données et les techniques innovantes d’analyse (Big data, IA, etc.)
Langue maternelle française et bonne maîtrise de la langue anglaise
Personne dynamique, autonome, rigoureuse, ouverte d'esprit, apte à travailler en équipe et ayant le sens de la communication --> com = présentation des résultats en conf


Langues
Langue maternelle française et bonne maîtrise de la langue anglaise
\end{comment}

Après avoir surmonté les difficultés dues à l'arrive des mécanismes à quartz, 
l'industrie horologère Suisse en est sortie renforcée, avec parmi ses porte-drapeau Omega SA dont la confiance accordés par vos clients prouvent la qualité. 
Néamoins, nous vivons dans une époque de changements exceptionels et il est primordial de garder une longueur d'avance sur la concurrence. 
L'arrivé des infrastructures sophistiquées pour le traitment des données a ouvert des nouveaux moyens aux entreprises pour améliorer leurs procédures et 
mieux maîtriser leurs opérations.

Au cours de mon expérience de chercheur, j'ai analysé des données complexes (regarder quels types de données sont traités pour voir si c'est + du essilor ou l'Oléal ou recherche)
je les ai mises sous forme visuelles dans des articles scientifiques puis les ait présentées à mes pairs (good com)
Je suis passionné par l'analyse et le traitement de données, et de leur utilisation pour améliorer et innover dans les produits (l'Oréal ?)
Rechercher et développer des outils de caractérisation et d’aide à l’analyse des données de contrôle -->  automatisant l'analyse des tests de régression en développant une série de scripts Python qui ont accéléré la capacité de l'équipe à répondre aux bugs logiciels.
Participer à la mise en place des structures de données de contrôle des montres et des mouvements --> j'ai également automatisé la mise à disposition de gros volume de données à destination des équipes de data scientists pour leur exploitation via des algorithmes d’apprentissage automatique.
Je connais les outils ETL
Je sais programmer et SQL n'a aucun secret pour moi 

J'ai suivi ma formation dans le domaine des sciences physiques et l'exploitation de grandes quantités de données fait partie de mon expérience professionnelle 
depuis longtemps. Tout au long de ma carrière de chercheur, j'ai utilisé des simulations numériques pour clarifier et prédire les proprietés de matériaux pour 
le stockage et la génération d'énergie. Le fruit de cette étude fut la communication des résultats avec la communauté de recherche par le biais 
d'articles scientifiques souvent rédigés en collaboration avec différentes équipes ou lors de conférences internationales au court desquelles 
\underline{j'ai eu la chance de présenter mes résultats}. Vous trouverez mes articles publiés dans des journaux scientifiques sur ma page Google Scholar (lien ci-dessus).
Mon rôle en tant que chercheur consistait à \underline{mettre en œuvre des algorithmes} pour modéliser les matériaux de batteries ou de matériaux photosensibles, 
afin d'effectuer des simulations et de collecter, transformer et finalement interpréter les résultats dans le contexte de leur signification physico-chimique. 
\begin{comment}
After having surmounted the difficulties due to the advent of quartz timepieces, the Swiss watchmaking industry came out stronger than before and with Omega SA as one of its standardbearers due to the confidence that customers have in the quality of its products. Nevertheless, we live in times of rapid change and keeping an edge over the competition is more important than ever. The advent of sofisticated data infrastructures provides companies with a means to improve processes and get a better handle of operations.  
My educational background is in the physical sciences and handling large amounts of data has been part of my professional experience for a long time. Throughout my career as a research scientist I relied on the power of computers and the data that they generate in order to elucidate and predict the properties of materials for energy storage and generation. The crown achievement of this investigative process, was the communication of the results with the research community by way of scientific articles often written in collaboration with different teams or in international conferences where I presented my results. My published articles can be found on my google scholar page (link above).
My role as a researcher consisted of implementing algorithms to model battery or solar cell materials, then carrying out simulations from which data could be gathered, transformed and ultimatley interpreted within the context of its physical/chemical meaning. The final aspect, and crown achievement of this investigative process, was the communication of the results with the research community by way of scientific articles often written in collaboration with different teams or in international conferences where I presented my results. My published articles can be found on my google scholar page (link above).
\end{comment}

Since leaving the world of academic research I have been able to adapt my skills to the fields of Data Engineering and Data Analysis. 
I have also extensive experience with Python, which I have used in several contexts, for example as a tool for analyzing data with 
\texttt{pandas}/\texttt{scikit-learn}, as a tool for moving/extracting data with \texttt{sqlalchemy} and as a means to visualize data with 
\texttt{matplotlib/seaborn/Plotly}, among others. Through practice, I have gained significant knowledge of SQL and tools like Dataiku DSS, 
for which I am an instructor.
Among my most notable professional experiences I count the time that I spend at Essilor. My job consisted of \underline{creating ETLs} written in python which 
handled data from the company's fabrication machines and optical calculators. These data were then made available through the Amazon cloud and analyzed 
using Dataiku DSS. My current project at L'Oréal has brought me \underline{to work on the construction of a data infrastructure} to handle the needs of the R\&D teams, 
thus it has been paramount to construct datawarehouses and datamarts to give users access to up to data information on things like worldwide products, ingredients, 
regulations, etc.  My added value in these projects appart from my daily activities has been twofold: by applying my data analysis skills I have been able to 
provide decision makers with valuable information with regard to the tasks that need to be prioritized while my capacity to abstract concepts has allowed me to 
automate time-consuming tasks.

\begin{comment}
I am a highly motivated person who is able to adapt to different environments, whether they be technical or cultural, having lived in Colombia, Ireland and France. The projects that I have worked on outside fundamental research have touched upon a theme that is common for many companies nowdays, namely the migration of their existing on-premise infrastructures towards the cloud. 
Leveraging the incredible value that is present within data will be vital for businesses in the coming years as it will allow them gain a competitive edge over their competitors. These are indeed exciting times for those of us who are working in this field due to the fast-paced evolution that we are witnessing. It is within this context that my eagerness to learn and stay up to date with the latest developments plays a critical role in distinguishing myself from other candidates. 
\end{comment}
Je suis une personne très motivée qui sait s'adapter à différents environnements, qu'ils soient techniques ou culturels. En effet, j'ai vécu en Colombie, en Irlande et en France et mon ouverture d'esprit ainsi que mon esprit d'équipe m'ont permi de travailler . 
Tirer parti de l'incroyable valeur ajoutée, possible grace à la bonne utilisation des données sera vital pour les entreprises dans les années à venir, pour faire face à leurs concurrents. 
Ce sont en effet des moments passionnants pour ceux d'entre nous qui travaillent dans ce domaine en raison de l'évolution rapide à laquelle nous assistons. 
C'est dans ce contexte que mon désir d'apprendre et de me tenir au courant des derniers développements joue un rôle essentiel pour me distinguer des autres candidats.
\
\end{cvletter}

%-------------------------------------------------------------------------------
% Print the signature and enclosures with above letter information
\makeletterclosing


\end{document}
