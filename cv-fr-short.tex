%%%%%%%%%%%%%%%%%
% This is an sample CV template created using altacv.cls
% (v1.1.5, 1 December 2018) written by LianTze Lim (liantze@gmail.com). Now compiles with pdfLaTeX, XeLaTeX and LuaLaTeX.
%
%% It may be distributed and/or modified under the
%% conditions of the LaTeX Project Public License, either version 1.3
%% of this license or (at your option) any later version.
%% The latest version of this license is in
%%    http://www.latex-project.org/lppl.txt
%% and version 1.3 or later is part of all distributions of LaTeX
%% version 2003/12/01 or later.
%%%%%%%%%%%%%%%%

%% If you need to pass whatever options to xcolor
\PassOptionsToPackage{dvipsnames}{xcolor}

%% If you are using \orcid or academicons
%% icons, make sure you have the academicons
%% option here, and compile with XeLaTeX
%% or LuaLaTeX.
% \documentclass[10pt,a4paper,academicons]{altacv}

%% Use the "normalphoto" option if you want a normal photo instead of cropped to a circle
% \documentclass[10pt,a4paper,normalphoto]{altacv}

% \documentclass[10pt,a4paper,normalphoto,academicons]{altacv}
\documentclass[10pt,a4paper,ragged2e,academicons]{altacv}

\setdefaultlanguage{french}
\sethyphenation{french}{} % Add words between the {} to avoid them to be cut

%% AltaCV uses the fontawesome and academicon fonts
%% and packages.
%% See texdoc.net/pkg/fontawecome and http://texdoc.net/pkg/academicons for full list of symbols. You MUST compile with XeLaTeX or LuaLaTeX if you want to use academicons.

% Change the page layout if you need to
\geometry{%
left=1cm,%
right=8.8cm,%
marginparwidth=6.9cm,%
marginparsep=1cm,%
top=1.25cm,%
bottom=1.25cm}

% Change the font if you want to, depending on whether
% you're using pdflatex or xelatex/lualatex
\ifxetexorluatex
  % If using xelatex or lualatex:
  %\setmainfont{Ubuntu}
   \setmainfont{Lato}
\else
  % If using pdflatex:
  \usepackage[utf8]{inputenc}
  \usepackage[T1]{fontenc}
  \usepackage[default]{lato}
\fi

% Change the colours if you want to
\definecolor{Mulberry}{HTML}{72243D}
\definecolor{SlateGrey}{HTML}{2E2E2E}
\definecolor{LightGrey}{HTML}{666666}
\definecolor{GoodSamaritan}{HTML}{3C6382}
\colorlet{heading}{GoodSamaritan}
% \colorlet{heading}{Sepia}
% \colorlet{accent}{Mulberry}
\colorlet{accent}{GoodSamaritan}
\colorlet{emphasis}{SlateGrey}
\colorlet{body}{LightGrey}

% Change the bullets for itemize and rating marker
% for \cvskill if you want to
\renewcommand{\itemmarker}{{\small\textbullet}}
\renewcommand{\ratingmarker}{\faCircle}

%% sample.bib contains your publications
% \addbibresource{sample.bib}

\begin{document}
\name{Mario Burbano}
\tagline{Cloud et Data Ingénieur/Analyste}
\photo{3.4cm}{img/CVPhoto.png}
\personalinfo{%
  % Not all of these are required!
  % You can add your own with \printinfo{symbol}{detail}
  \printinfo{Date de naissance}{14 November 1984}
  \printinfo{Nationalité}{Irlandaise/Colombienne}
  \location{91 rue du Colonel Fabien, 92160 Antony, FRANCE}
  %\vspace{1ex}
  \printinfo{État civil}{En concubinage}
  %\vspace{1ex}
  \email{burbanom@tcd.ie}
  \phone{+ 33 6 43 27 79 22}
  % \mailaddress{My Address Here}
  % \twitter{@twitterhandle}
  \github{burbanom}
  \linkedin{burbanom}
}

%% Make the header extend all the way to the right, if you want.
\begin{fullwidth}
\makecvheader
\parbox{.7\paperwidth}{%
Docteur en chimie numérique, j'ai une grande expérience dans le Cloud Computing, l'automatisation des tâches et l'analyse de données. 
Je suis passionné par la mise en œuvre de systèmes de traitement de données et par la façon dont ils peuvent faciliter le processus de prise de décision.
}

\end{fullwidth}

%% Depending on your tastes, you may want to make fonts of itemize environments slightly smaller
\AtBeginEnvironment{itemize}{\small}

%% Provide the file name containing the sidebar contents as an optional parameter to \cvsection.
%% You can always just use \marginpar{...} if you do
%% not need to align the top of the contents to any
%% \cvsection title in the "main" bar.
% \marginpar{% \cvsection{My Life Philosophy}
%
% \begin{quote}
% ``Something smart or heartfelt, preferably in one sentence.''
% \end{quote}
%
% \cvsection{Most Proud of}
%
% \cvachievement{\faTrophy}{Fantastic Achievement}{and some details about it}
%
% \divider
%
% \cvachievement{\faHeartbeat}{Another achievement}{more details about it of course}
%
% \divider
%
% \cvachievement{\faHeartbeat}{Another achievement}{more details about it of course}

\cvsection{Skills}
\cvskill{High Performance Computing}{5}

\divider

\cvskill{Molecular Modelling}{5}

\divider

\cvskill{Mathematics / Statistics}{3}

\divider

\cvskill{Data Visualization}{4}

\divider

\cvsubsection{\color{accent}Computer science}

\cvskill{Python}{5}
\cvskill{Linux/Unix/Bash}{4}
\cvskill{Fortran}{4}
\cvskill{Machine Learning}{3}
\cvskill{git}{3}

\medskip

\cvtag{AWS}
\cvtag{GCP}
\cvtag{Plotly/Dash}
\cvtag{scikit-learn}
\cvtag{Flask}
\cvtag{pandas}
\cvtag{Visual Studio Code}

\medskip

\cvsubsection{\color{accent}Simulations code}

\cvtag{Lammps}
\cvtag{VASP}
\cvtag{Gromacs}
\cvtag{Amber}
\cvtag{Gaussian}
\cvtag{Orca}
\cvtag{VMD}

%\cvsubsection{Basic Knowledge}
% \cvtag{Hard-working}
% \cvtag{Eye for detail}
% \cvtag{Motivator \& Leader}

% \divider\smallskip
%
% \cvtag{C++}
% \cvtag{Embedded Systems}
% \cvtag{Statistical Analysis}

\cvsection{Languages}

\cvskill{English}{5}

\divider

\cvskill{Spanish}{5}

\divider

\cvskill{French}{4}

%% Yeah I didn't spend too much time making all the
%% spacing consistent... sorry. Use \smallskip, \medskip,
%% \bigskip, \vpsace etc to make ajustments.
\medskip

\cvsection{Education}

\cvevent{\small Ph.D.\ in Computational Chemistry}{}{2006 -- 2009}{Université Paris-Sud 11}
%Theoretical study of photophysics properties of fluorescent proteins

% \divider

\cvevent{\small M.Sc.\ in Physical-Chemistry}{}{2004 -- 2006}{Université Paris-Sud 11}

% \divider

% \cvevent{\small Magistère de Physico-Chimie Moléculaire}{}{2003 -- 2006}{Université Paris-Sud 11 ENS Cachan}
\cvevent{\small Magistère de Physico-Chimie Moléculaire}{}{2003 -- 2006}{Université Paris-Sud 11}
\vspace{-6pt}\hspace{3.8cm}{\color{body}\small ENS Cachan}
% \cvevent{\small Magistère de Physico-Chimie Moléculaire}{}{2003 -- 2006}{\shortstack{Université Paris-Sud 11\\ ENS Cachan}}

% \divider

% \cvevent{\small B.Sc. \& CPGE\ in Physical-Chemistry}{Université Paris-Sud 11 / Lycée F. Arago}{2001 -- 2003}{}
\vspace{3pt}
\cvevent{\small CPGE\ in Physics \& Chemistry}{}{2001 -- 2003}{Lycée F. Arago Perpignan}
}
% \cvsection{Experience}
\cvsection[sidebar-fr-0]{Expérience professionnelle}

\cvevent{Ingénieur Data}{Ysance/Devoteam}{2021 -- présent}{Île de France, France}
\begin{itemize}
\justifying
\item \textbf{L’Oréal} Au sein de l’équipe IT/BI du département R\&D, j'ai participé à un projet dont le but était de migrer l’infrastructure de traitement de données depuis l'environnement existant: Talend/Hadoop vers une solution Airflow/GCP.
\end{itemize}

\divider

\cvevent{Ingénieur/Analyste et formateur}{Lincoln/Alten}{2019 -- 2020}{Île de France, France}
\begin{itemize}
\justifying
\item \textbf{Malakoff Humanis} Au sein de l'équipe en charge de l'infrastructure de données, j'ai développé une série de scripts visant à analyser les données nécessaires à la réussite de la migration des projets d'apprentissage automatique de l'entreprise développés sur Dataiku DSS. Ces modèles reposaient sur des données hébergées sur site qui devaient être déplacées vers le cloud AWS.
\item \textbf{Orange} J'ai intégré l'équipe Marketing Grand Public afin de migrer les datamarts SAS existants vers Dataiku DSS. J'ai ensuite formé plusieurs équipes à ce nouvel outil.\end{itemize}

\divider

\cvevent{Ingénieur/Analyste Data}{Altran}{2018 -- 2019}{Île de France, France}
\begin{itemize}
\justifying  
\item \textbf{Essilor} En tant que membre de l'équipe chargée d'implémenter et de maintenir le logiciel utilisé en interne pour les calculs optiques, j'ai participé à la poussée vers la création d'une infrastructure de données sur le cloud AWS. L'objectif était de pouvoir exploiter les données en les mettant à la disposition des équipes data science et R\&D grâce à l'utilisation d'ETL et de l'infrastructure de données.
\item \textbf{Essilor} J'ai ensuite facilité le travail de mon équipe en automatisant l'analyse des tests de régression à l'aide d'une série de scripts Python qui ont accéléré la capacité de l'équipe à répondre aux bugs logiciels.
\end{itemize}

\divider

\cvevent{Ingénieur de recherche}{CEA}{2016 -- 2018}{Saclay, France}
\begin{itemize}
\justifying  
\item Réusinage de code FORTRAN/MPI:
Au sein de l'équipe Simulation, j'ai réécrit et créé des modules pour un programme de simulation électrochimique.  
Ce logiciel est un code de Dynamique Moléculaire qui permet de simuler des \textit{supercapaciteurs} à potentiel constant.
\item Dans le cadre de cette intervention, j'ai formulé une nouvelle méthode de résolution des équations d’électrostatique afin de rendre le code plus performant. 
\end{itemize}

\clearpage
\cvsection[sidebar-fr-1]{Expérience professionnelle -- suite}

\cvevent{Chercheur postdoctoral}{UPMC/CNRS}{2014 -- 2016}{Paris, France}
\begin{itemize}
\justifying  
\item Dans le contexte de ce postdoctorat, j'ai utilisé Python pour ajuster les paramètres d'un modèle d'interactions ioniques afin d'étudier les mouvements corrélés au sein des électrolytes solides pour batteries lithium-ion.
\item J'ai également établi des procédures pour générer et analyser de grandes quantités de données en FORTRAN/Python, ensuite utilisées pour expliquer les propriétés des matériaux de ce type de batterie.
\end{itemize}

\divider

\cvevent{Doctorat en Chimie Numérique}{Trinity College Dublin}{2009 -- 2013}{Dublin, Ireland}
Modélisation numérique des oxydes métalliques
\smallskip
\begin{itemize}
\justifying  
\item Dans le contexte de mon Doctorat, j'ai réalisé des simulations moléculaires de matériaux spécifiques à la production et au stockage d'énergie.
\item Mes recherches ont permises, grâce à l'utilisation de prédictions théoriques, de dissiper les idées fausses concernant le rôle des impuretés et de la morphologie de ces matériaux sur leurs proprietés. 
\item Pour cela, j'ai utilisé Fortran/MPI/Python pour coder des programmes de simulation et d'analyse de données.
\end{itemize}
{\small 12 articles scientifiques évalués par des pairs, h-\textit{index} 11, 577 citations aussi que plusieurs conférences internationales.}

%% If the NEXT page doesn't start with a \cvsection but you'd
%% still like to add a sidebar, then use this command on THIS
%% page to add it. The optional argument lets you pull up the
%% sidebar a bit so that it looks aligned with the top of the
%% main column.
% \addnextpagesidebar[-1ex]{page3sidebar}

\end{document}