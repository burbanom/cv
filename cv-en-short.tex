%%%%%%%%%%%%%%%%%
% This is an sample CV template created using altacv.cls
% (v1.1.5, 1 December 2018) written by LianTze Lim (liantze@gmail.com). Now compiles with pdfLaTeX, XeLaTeX and LuaLaTeX.
%
%% It may be distributed and/or modified under the
%% conditions of the LaTeX Project Public License, either version 1.3
%% of this license or (at your option) any later version.
%% The latest version of this license is in
%%    http://www.latex-project.org/lppl.txt
%% and version 1.3 or later is part of all distributions of LaTeX
%% version 2003/12/01 or later.
%%%%%%%%%%%%%%%%

%% If you need to pass whatever options to xcolor
\PassOptionsToPackage{dvipsnames}{xcolor}

%% If you are using \orcid or academicons
%% icons, make sure you have the academicons
%% option here, and compile with XeLaTeX
%% or LuaLaTeX.
% \documentclass[10pt,a4paper,academicons]{altacv}

%% Use the "normalphoto" option if you want a normal photo instead of cropped to a circle
% \documentclass[10pt,a4paper,normalphoto]{altacv}

% \documentclass[10pt,a4paper,normalphoto,academicons]{altacv}
\documentclass[10pt,a4paper,ragged2e,academicons]{altacv}

\setdefaultlanguage[variant=british]{english}
\sethyphenation{english}{} % Add words between the {} to avoid them to be cut

%% AltaCV uses the fontawesome and academicon fonts
%% and packages.
%% See texdoc.net/pkg/fontawecome and http://texdoc.net/pkg/academicons for full list of symbols. You MUST compile with XeLaTeX or LuaLaTeX if you want to use academicons.

% Change the page layout if you need to
\geometry{%
left=1cm,%
right=8.8cm,%
marginparwidth=6.9cm,%
marginparsep=1cm,%
top=1.25cm,%
bottom=1.25cm}

% Change the font if you want to, depending on whether
% you're using pdflatex or xelatex/lualatex
\ifxetexorluatex
  % If using xelatex or lualatex:
  %\setmainfont{Ubuntu}
   \setmainfont{Lato}
\else
  % If using pdflatex:
  \usepackage[utf8]{inputenc}
  \usepackage[T1]{fontenc}
  \usepackage[default]{lato}
\fi

% Change the colours if you want to
\definecolor{Mulberry}{HTML}{72243D}
\definecolor{SlateGrey}{HTML}{2E2E2E}
\definecolor{LightGrey}{HTML}{666666}
\definecolor{GoodSamaritan}{HTML}{3C6382}
\colorlet{heading}{GoodSamaritan}
% \colorlet{heading}{Sepia}
% \colorlet{accent}{Mulberry}
\colorlet{accent}{GoodSamaritan}
\colorlet{emphasis}{SlateGrey}
\colorlet{body}{LightGrey}

% Change the bullets for itemize and rating marker
% for \cvskill if you want to
\renewcommand{\itemmarker}{{\small\textbullet}}
\renewcommand{\ratingmarker}{\faCircle}

%% sample.bib contains your publications
% \addbibresource{sample.bib}

\begin{document}
\name{Mario Burbano}
\tagline{Cloud Data Engineer and Analyst}
\photo{3.4cm}{img/MB.png}
\personalinfo{%
  % Not all of these are required!
  % You can add your own with \printinfo{symbol}{detail}
  \printinfo{Date of birth}{14 November 1984}
  \printinfo{Nationality}{Irish/Colombian}
  \location{91 rue du Colonel Fabien, 92160 Antony, FRANCE}
  %\vspace{1ex}
  \printinfo{Marital status}{In cohabitation}
  %\vspace{1ex}
  \email{burbanom@tcd.ie}
  \phone{+ 33 6 43 27 79 22}
  % \mailaddress{My Address Here}
  % \twitter{@twitterhandle}
  \github{burbanom}
  \linkedin{burbanom}
}

%% Make the header extend all the way to the right, if you want.
\begin{fullwidth}
\makecvheader
\parbox{.7\paperwidth}{%
PhD in Computational Chemistry with extensive experience in cloud computing, task automation and data analysis. I am passionate about the implementation of data pipelines and how they can facilitate the process of data-driven decision-making.
}

\end{fullwidth}

%% Depending on your tastes, you may want to make fonts of itemize environments slightly smaller
\AtBeginEnvironment{itemize}{\small}

%% Provide the file name containing the sidebar contents as an optional parameter to \cvsection.
%% You can always just use \marginpar{...} if you do
%% not need to align the top of the contents to any
%% \cvsection title in the "main" bar.
% \marginpar{% \cvsection{My Life Philosophy}
%
% \begin{quote}
% ``Something smart or heartfelt, preferably in one sentence.''
% \end{quote}
%
% \cvsection{Most Proud of}
%
% \cvachievement{\faTrophy}{Fantastic Achievement}{and some details about it}
%
% \divider
%
% \cvachievement{\faHeartbeat}{Another achievement}{more details about it of course}
%
% \divider
%
% \cvachievement{\faHeartbeat}{Another achievement}{more details about it of course}

\cvsection{Skills}
\cvskill{High Performance Computing}{5}

\divider

\cvskill{Molecular Modelling}{5}

\divider

\cvskill{Mathematics / Statistics}{3}

\divider

\cvskill{Data Visualization}{4}

\divider

\cvsubsection{\color{accent}Computer science}

\cvskill{Python}{5}
\cvskill{Linux/Unix/Bash}{4}
\cvskill{Fortran}{4}
\cvskill{Machine Learning}{3}
\cvskill{git}{3}

\medskip

\cvtag{AWS}
\cvtag{GCP}
\cvtag{Plotly/Dash}
\cvtag{scikit-learn}
\cvtag{Flask}
\cvtag{pandas}
\cvtag{Visual Studio Code}

\medskip

\cvsubsection{\color{accent}Simulations code}

\cvtag{Lammps}
\cvtag{VASP}
\cvtag{Gromacs}
\cvtag{Amber}
\cvtag{Gaussian}
\cvtag{Orca}
\cvtag{VMD}

%\cvsubsection{Basic Knowledge}
% \cvtag{Hard-working}
% \cvtag{Eye for detail}
% \cvtag{Motivator \& Leader}

% \divider\smallskip
%
% \cvtag{C++}
% \cvtag{Embedded Systems}
% \cvtag{Statistical Analysis}

\cvsection{Languages}

\cvskill{English}{5}

\divider

\cvskill{Spanish}{5}

\divider

\cvskill{French}{4}

%% Yeah I didn't spend too much time making all the
%% spacing consistent... sorry. Use \smallskip, \medskip,
%% \bigskip, \vpsace etc to make ajustments.
\medskip

\cvsection{Education}

\cvevent{\small Ph.D.\ in Computational Chemistry}{}{2006 -- 2009}{Université Paris-Sud 11}
%Theoretical study of photophysics properties of fluorescent proteins

% \divider

\cvevent{\small M.Sc.\ in Physical-Chemistry}{}{2004 -- 2006}{Université Paris-Sud 11}

% \divider

% \cvevent{\small Magistère de Physico-Chimie Moléculaire}{}{2003 -- 2006}{Université Paris-Sud 11 ENS Cachan}
\cvevent{\small Magistère de Physico-Chimie Moléculaire}{}{2003 -- 2006}{Université Paris-Sud 11}
\vspace{-6pt}\hspace{3.8cm}{\color{body}\small ENS Cachan}
% \cvevent{\small Magistère de Physico-Chimie Moléculaire}{}{2003 -- 2006}{\shortstack{Université Paris-Sud 11\\ ENS Cachan}}

% \divider

% \cvevent{\small B.Sc. \& CPGE\ in Physical-Chemistry}{Université Paris-Sud 11 / Lycée F. Arago}{2001 -- 2003}{}
\vspace{3pt}
\cvevent{\small CPGE\ in Physics \& Chemistry}{}{2001 -- 2003}{Lycée F. Arago Perpignan}
}
% \cvsection{Experience}
\cvsection[sidebar-en-0]{Professional experience}

\cvevent{Data Engineer}{Ysance/Devoteam}{2021 -- Ongoing}{Île de France, France}
\begin{itemize}
\item \textbf{L'Oréal} As part of one of the IT/BI teams within the R\&D deparment, I participated in a project whose aim was to migrate the existing data pipelines from a Talend/Hadoop environement towards an Airflow/GCP solution.
\end{itemize}

\divider

\cvevent{Data Engineer/Scientist and instructor}{Lincoln/Alten}{2019 -- 2020}{Île de France, France}
\begin{itemize}
\item \textbf{Malakoff Humanis} As part of the team in charge of the data infrastructure, I developed a series of scripts aimed at analyzing the data required for the successful migration of the company's machine learning projects developed on Dataiku DSS. These models relied on data hosted on-premise which was to be moved to the AWS cloud. 
\item \textbf{Orange} I integrated the General Public Marketing team in order to migrate the existing SAS datamarts to Dataiku DSS. I also provided several teams with training for this new tool. 
\end{itemize}

\divider

\cvevent{Data Engineer/Analyst}{Altran}{2018 -- 2019}{Île de France, France}
\begin{itemize}
\item \textbf{Essilor} As a member of the team tasked with implementing and maintaining the software used internally for optical calculations, I participated in the push towards the creation of a data infrastructure on the cloud AWS. The aim was to be able to exploit the data by making it available to the data science and R\&D teams. I also contributed to the team by automating the analysis of regression tests by developing a series of Python scripts which accelerated the team's ability to respond to software bugs.
\end{itemize}

\divider

\cvevent{Research Engineer}{CEA}{2016 -- 2018}{Saclay, France}
\begin{itemize}
\item I carried out refactoring and modularisation of an electrochemistry modelling program used to perform Molecular Dynamics simulations of \textit{supercapacitors} at constant potential.
\item I implemented a new method for solving electrostatic equations, which was then made available as a stand-alone Fortran library.  
\end{itemize}

\clearpage
\cvsection[sidebar-en-1]{Professional experience -- cont.}

\cvevent{Postdoctoral researcher}{UPMC}{2014 -- 2016}{Paris, France}
\begin{itemize}
\item Using Python, I fitted models to study correlated motion in
battery components.
\item I established procedures to generate/analyse large
quantities of data used to explain materials'
properties using Fortran/Python. 
\end{itemize}

\divider

\cvevent{Ph.D. in Computational Chemistry}{Trinity College Dublin}{2009 -- 2013}{Dublin, Ireland}
Computer modelling of metal oxides
\smallskip
\begin{itemize}
\item I performed molecular simulations of materials for energy production and storage
\item Using theoretical predictions, I helped dispell
misconceptions regarding the roles of impurities
and morphology as possible enhancers of
desired qualities in materials used to generate
energy.
\item I used Fortran/MPI to write simulation and data analysis programs
\end{itemize}
{\small 12 peer-reviewed articles, h-\textit{index} 11, 577 citations}

%% If the NEXT page doesn't start with a \cvsection but you'd
%% still like to add a sidebar, then use this command on THIS
%% page to add it. The optional argument lets you pull up the
%% sidebar a bit so that it looks aligned with the top of the
%% main column.
% \addnextpagesidebar[-1ex]{page3sidebar}

\end{document}